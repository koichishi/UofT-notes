\documentclass[11pt]{article}

% Libraries.
\usepackage{amsmath}
\usepackage{amssymb}
\usepackage{pgfplots}
\usepackage{graphicx}
\usepackage{enumitem}
\usepackage{hyperref}
\usepackage{fancyhdr}
\usepackage{perpage}
\usepackage{float}

% Property settings.
\MakePerPage{footnote}
\pagestyle{fancy}
\lhead{Notes by Y.W.}

% Commands
\newcommand{\ti}[1]{\textit{#1}}
\newcommand{\tb}[1]{\textbf{#1}}
\newcommand{\mb}[1]{\mathbb{#1}}
\newcommand{\under}[1]{\underline{#1}}
\newcommand{\proof}[0]{\textit{\underline{proof: }}}
\newcommand{\real}[0]{\mathbb{R}}
\newcommand{\va}[0]{\tb{a}}
\newcommand{\vb}[0]{\tb{b}}
\newcommand{\vi}[0]{\tb{i}}
\newcommand{\vj}[0]{\tb{j}}
\newcommand{\vk}[0]{\tb{k}}
% Attr.
\title{Math Notes}
\author{Yuchen Wang}
\date{\today}

\begin{document}
	\maketitle
	\tableofcontents
	\newpage
\section{Hyperbolic Functions} 
$$\sinh(x) = \frac{e^x - e^{-x}}{2}$$
$$\cosh(x) = \frac{e^x + e^{-x}}{2}$$
$$\tanh(x) = \frac{\sinh(x)}{\cosh(x)} = \frac{e^x - e^{-x}}{e^x + e^{-x}} = \frac{e^{2x} - 1}{e^{2x} + 1}$$
\section{Trigonometric Formulas}
$$\tan^2{\theta} + 1 = \sec^2{\theta}$$
$$1 + \cot^2{\theta} = \csc^2{\theta}$$
$$\cos{(a+b)} = \cos{a}\cos{b}-\sin{a}\sin{b}$$
$$\sin{(a+b)} = \sin{a}\cos{b}+\cos{a}\sin{b}$$
$$\cos^2{a} = \frac{1+\cos{2a}}{2} $$
$$\sin^2{a} = \frac{1-\cos{2a}}{2} $$
\section{Arc functions}
\begin{center}
\begin{tabular}{ c c c c c}
 Name & Usual notation & Definition & Domain & Range \\ 
 \tb{arcsine} & $y = \arcsin(x)$ & $x = \sin(y)$ & $[-1, 1]$ & $[-\frac{\pi}{2}, \frac{\pi}{2}] $ \\  
 \tb{arccosine} & $y = \arccos(x)$ & $x = \cos(y)$ & $[-1, 1]$ & $[0, \pi]$   \\
 \tb{arctangent} & $y = \arctan(x)$ & $x = \tan(y)$ & $\real$ & $(-\frac{\pi}{2}, \frac{\pi}{2})$   \\
\end{tabular}
\end{center}

\section{Cross Product}
\paragraph{Definition} In 3-dimensional Euclidean space only, the cross product of vectors \tb{a} and \tb{b} is 
$$\tb{a} \times \tb{b} = \begin{pmatrix}a_2b_3 - a_3b_2\\ a_3b_1-a_1b_3\\a_1b_2-a_2b_1\end{pmatrix}$$ 
\paragraph{Remark} "xia, dafan, shang"
\paragraph{As a Determinant}
\begin{align*}
	\va \times \vb &= \begin{vmatrix}
		\vi & \vj & \vk \\
		a_1 & a_2 & a_3 \\
		b_1 & b_2 & b_3
	\end{vmatrix} \\
	&= \begin{vmatrix}
		a_2 & a_3 \\
		b_2 & b_3 \\
	\end{vmatrix} \vi - \begin{vmatrix}
		a_1 & a_3 \\
		b_1 & b_3 \\
	\end{vmatrix} \vj +  \begin{vmatrix}
		a_1 & a_2 \\
		b_1 & b_2 \\
	\end{vmatrix} \vk
\end{align*}
\paragraph{Properties}
\begin{enumerate}
	\item $\tb{a} \times \tb{b}$ is orthogonal to both \tb{a} and \tb{b}
	\item $|\tb{a} \times \tb{b}| = |\tb{a}||\tb{b}|sin\theta$. This says that the length $\tb{a} \times \tb{b}$ equals the area of the parallelogram generated by \tb{a} and \tb{b}.
	\item $\tb{a} \times \tb{b} = -\tb{b} \times \tb{a}$
	\item $(c_1\tb{a}_1 + c_2\tb{a}_2) \times \tb{b} = c_1\tb{a}_1 \times \tb{b} + c_2\tb{a}_2 \times \tb{b}$
	\item $\tb{i} \times \tb{j} = \tb{k}$ and $\tb{j} \times \tb{k} = \tb{i}$ and $\tb{k} \times \tb{i} = \tb{j}$
	\item $Not$ associative: $(a \times b) \times c \neq a \times (b \times c)$
\end{enumerate} 
\section{Derivative of Logarithmic Functions}
$$\frac{d}{dx}\log_a{x} = \frac{1}{x \cdot ln(a)}$$

\section{Common Taylor Series}
\begin{align}
	e^x &= \sum_{n=0}^{\infty}\frac{x^n}{n!}\\
	\sin{x} &= \sum_{n=0}^{\infty}(-1)^n\frac{x^{2n+1}}{(2n+1)!}\\
	\cos{x} &= \sum_{n=0}^{\infty}(-1)^n\frac{x^{2n}}{(2n)!}\\
	\frac{1}{1-x} &= \sum_{n=0}^{\infty}x^n
\end{align}
\paragraph{Remark}
Take the primitive of (4) to get the Taylor polynomial of $\ln(1-x)$.
\section{$\varepsilon$ definition of supremum and infimum}
\paragraph{Definition} Let S be a nonempty subset of the real numbers that is bounded above. The upper bound $u$ is said to be the supremum of S iff $$\forall \varepsilon >0, \exists x \in S, u - \varepsilon < x$$
\paragraph{Definition} Let S be a nonempty subset of the real numbers that is bounded below. The lower bound $w$ is said to be the infimum of S iff $$\forall \varepsilon >0, \exists x \in S, x < w + \varepsilon$$
\section{Even and Odd functions}
\subsection{Profucts}
The product of two even functions is an even function. \\
The product of two odd functions is an even function. \\
The product of an even function and an odd function is an odd function.\\
The quotient of two even functions is an even function.
\subsection{Odd-Even Decomposition}
For any continuous function $f$, $f$ can be decomposed into the sum of one even function and one odd function:
$$f(x) = \frac{1}{2}(f(x) + f(-x)) + \frac{1}{2}(f(x) - f(-x))$$
We can verify that the first part is even, and the second part is odd.
\section{Real roots and Polynomials}
Every polynomial $p(x)$ of degree 3 with real coefficients has at least one real root. For $x = A$ sufficiently negative, $p(x) < 0$; for $x = B$ sufficiently positive, $p(x) > 0$, since the degree is odd. Hence, by the intermediate value theorem for continuous functions, there is at least one solution $x_0$ of $p(x) = 0$ between A and B.
\section{Bilinear Map}
A \tb{bilinear map} is a function combining elements of two vector spaces to yield an element of a third vector space, and is linear in each of its arguments. \\
In other words, when we hold the first entry of the bilinear map fixed while letting the second entry vary, the result is a linear operator, and similarly for when we hold the second entry fixed.(example: matrix multiplication)
\end{document}
