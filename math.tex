\documentclass[11pt]{article}

% Libraries.
\usepackage{amsmath}
\usepackage{amssymb}
\usepackage{pgfplots}
\usepackage{graphicx}
\usepackage{enumitem}
\usepackage{hyperref}
\usepackage{fancyhdr}
\usepackage{perpage}
\usepackage{float}

% Property settings.
\MakePerPage{footnote}
\pagestyle{fancy}
\lhead{Notes by Y.W.}

% Commands
\newcommand{\ti}[1]{\textit{#1}}
\newcommand{\tb}[1]{\textbf{#1}}
\newcommand{\mb}[1]{\mathbb{#1}}
\newcommand{\under}[1]{\underline{#1}}
\newcommand{\proof}[0]{\textit{\underline{proof: }}}
\newcommand{\real}[0]{\mathbb{R}}
% Attr.
\title{Math Notes}
\author{Yuchen Wang}
\date{\today}

\begin{document}
	\maketitle
	\tableofcontents
	\newpage
\section{Hyperbolic Functions} 
$$sinh(x) = \frac{e^x - e^{-x}}{2}$$
$$cosh(x) = \frac{e^x + e^{-x}}{2}$$
\section{Trigonometric Formulas}
$$\tan^2{\theta} + 1 = \sec^2{\theta}$$
$$1 + \cot^2{\theta} = \csc^2{\theta}$$
$$\cos{(a+b)} = \cos{a}\cos{b}-\sin{a}\sin{b}$$
$$\sin{(a+b)} = \sin{a}\cos{b}+\cos{a}\sin{b}$$
$$\cos^2{a} = \frac{1+\cos{2a}}{2} $$
$$\sin^2{a} = \frac{1-\cos{2a}}{2} $$
\section{Arc functions}
\begin{center}
\begin{tabular}{ c c c c c}
 Name & Usual notation & Definition & Domain & Range \\ 
 \tb{arcsine} & $y = \arcsin(x)$ & $x = \sin(y)$ & $[-1, 1]$ & $[-\frac{\pi}{2}, \frac{\pi}{2}] $ \\  
 \tb{arccosine} & $y = \arccos(x)$ & $x = \cos(y)$ & $[-1, 1]$ & $[0, \pi]$   \\
 \tb{arctangent} & $y = \arctan(x)$ & $x = \tan(y)$ & $\real$ & $(-\frac{\pi}{2}, \frac{\pi}{2})$   \\
\end{tabular}
\end{center}

\section{Cross Product}
\paragraph{Definition} In 3-dimensional Euclidean space only, the cross product of vectors \tb{a} and \tb{b} is 
$$\tb{a} \times \tb{b} = \begin{pmatrix}a_2b_3 - a_3b_2\\ a_3b_1-a_1b_3\\a_1b_2-a_2b_1\end{pmatrix}$$ 
\paragraph{Remark} "xia, dafan, shang"
\paragraph{Properties}
\begin{enumerate}
	\item $\tb{a} \times \tb{b}$ is orthogonal to both \tb{a} and \tb{b}
	\item $|\tb{a} \times \tb{b}| = |\tb{a}||\tb{b}|sin\theta$. This says that the length $\tb{a} \times \tb{b}$ equals the area of the parallelogram generated by \tb{a} and \tb{b}.
	\item $\tb{a} \times \tb{b} = -\tb{b} \times \tb{a}$
	\item $(c_1\tb{a}_1 + c_2\tb{a}_2) \times \tb{b} = c_1\tb{a}_1 \times \tb{b} + c_2\tb{a}_2 \times \tb{b}$
	\item $\tb{i} \times \tb{j} = \tb{k}$ and $\tb{j} \times \tb{k} = \tb{i}$ and $\tb{k} \times \tb{i} = \tb{j}$
	\item $Not$ associative: $(a \times b) \times c \neq a \times (b \times c)$
\end{enumerate} 
\section{Derivative of Logarithmic Functions}
$$\frac{d}{dx}\log_a{x} = \frac{1}{x \cdot ln(a)}$$

\section{Common Taylor Series}
\begin{align}
	e^x &= \sum_{n=0}^{\infty}\frac{x^n}{n!}\\
	\sin{x} &= \sum_{n=0}^{\infty}(-1)^n\frac{x^{2n+1}}{(2n+1)!}\\
	\cos{x} &= \sum_{n=0}^{\infty}(-1)^n\frac{x^{2n}}{(2n)!}\\
	\frac{1}{1-x} &= \sum_{n=0}^{\infty}x^n
\end{align}
\paragraph{Remark}
Take the primitive of (4) to get the Taylor polynomial of $\ln(1-x)$.
\section{$\varepsilon$ definition of supremum and infimum}
\paragraph{Definition} Let S be a nonempty subset of the real numbers that is bounded above. The upper bound $u$ is said to be the supremum of S iff $$\forall \varepsilon >0, \exists x \in S, u - \varepsilon < x$$
\paragraph{Definition} Let S be a nonempty subset of the real numbers that is bounded below. The lower bound $w$ is said to be the infimum of S iff $$\forall \varepsilon >0, \exists x \in S, x < w + \varepsilon$$
\end{document}
