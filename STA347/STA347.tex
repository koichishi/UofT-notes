\documentclass[11pt]{article}
% Libraries.
\usepackage{amsmath}
\usepackage{amssymb}
\usepackage{pgfplots}
\usepackage{graphicx}
\usepackage{enumitem}
\usepackage{hyperref}
\usepackage{fancyhdr}
\usepackage{perpage}
\usepackage{float}
\usepackage{esint}
\usepackage[margin=1.5cm]{geometry}
\usepackage{../raina}

% Property settings.
\MakePerPage{footnote}
\pagestyle{headings}

% Attr.
\title{STA347 \\ Final Preparation}
\author{Yuchen Wang}
\date{\today}

\begin{document}
    \maketitle
    \tableofcontents
    \newpage
    \section{Experiments, Events and Sample Spaces}
    \definition{Experiment, Sample space and event}
    \begin{itemize}
    	\item Experiment: Any process, real or hypothetical, in which the possible outcomes can be identified ahead of time;
    	\item Sample space: The collection of all possible outcomes, denoted by $S$;
    	\item Event: A well-defined subset of sample space
    \end{itemize}
   \definition[countably infinity]
   A set is \tb{countably infinite} if its elements can be put in one-to-one correspondence with the set of natural numbers.
   \definition[At most countable sets]
   A set that is either finite or countably infinite is called an \tb{at most countable set}.
   \theorem
   Suppose $E, E_1, E_2, \hdots$ are events. The following are also events
   \begin{enumerate}
   	\item $E^c$
   	\item $E_1 \cup E_2 \cup \hdots E_n$
   	\item $\sum_{i=1}^\infty E_i$
   \end{enumerate}
   \section{Definition and Properties of Probability}
   \definition[$\sigma$-field]
   Let $\chi$ be a space. A collection $\mc{F}$ of subsets of $\chi$ is called a \tb{$\sigma$-field} if
   \begin{enumerate}
   	\item $\chi \in \mc{F}$
   	\item (closure under complement) if $E \in \mc{F}$, then $E^c \in \mc{F}$
   	\item (closure under countable union) if $E_1, E_2, \hdots \in \mc{F}$, then $\cup_{n=1}^\infty E_n \in \mc{F}$
   \end{enumerate}
   \remark A $\sigma$-field refers to the collection of subsets of a sample space that we should use in order to establish a mathematically formal definition of probability. The sets in the $\sigma$-field constitute the events from our sample space.
   \axiom[Axioms of Probability]
   Let $S$ be a sample space, and let $\mc{F}$ be a $\sigma$-field of $S$.
   \begin{itemize}
   	\item Axiom 1 (non-negativity) \(P(E) \geq 0\) for any event \(E \in \mathcal{F}\).
   	\item Axiom \(2 \quad P(S)=1\)
   	\item Axiom 3 (countable additivity) For every sequence of disjoint events \(E_{1}, E_{2}, \ldots \in \mathcal{F}\)
$$
P\left(\stackrel{\infty}{\cup} E_{i}\right)=\sum_{i=1}^{\infty} P\left(E_{i}\right)
$$
   \end{itemize}
   \definition[probability] Any function $P$ on a sample space $S$ satisfying Axioms 1-3 is called a \tb{probability}.
   \definition[disjoint sets] Sets $A$ and $B$ are \tb{disjoint} if $A \cap B = \emptyset$.
   \theorem{Properties of Probability}
   \begin{enumerate}
   	\item $P(\emptyset) = 0$
   	\item (finite additivity) For any disjoint events $E_1, \hdots, E_n$, $$P(\cup_{i=1}^n E_i) = \sum_{i=1}^n P(E_i)$$
   	\item $P(A^c) = 1 - P(A)$
   	\item For $A \subset B, P(A) \leq P(B)$
   	\item $0 \leq P(A) \leq 1$
   	\item $P(A- B) = P(A) - P(A \cap B)$
   	\item $P(A\cup B) = P(A) + P(B) - P(A \cap B)$
   	\item (subadditivity, Boole's inequality) For any events $E_1, \hdots, E_n$,
   	$$P(\cup_{i=1}^n E_i) \leq \sum_{i=1}^n P(E_i)$$
   \end{enumerate}

\theorem[Continuity from below and above] Let $P$ be a probability. \\
(continuity from below) If $A_n \nearrow A$ (i.e. $A_1 \subset A_2 \subset \hdots$ and $\cup_n A_n = A$), then $P(A_n) \nearrow P(A)$ \\
(continuity from above) If $A_n \searrow A$ (i.e. $A_1 \supset A_2 \supset \hdots$ and $\cap_n A_n = A$), then $P(A_n) \searrow P(A)$ \\

\subsection{Finite Sample Spaces}
Suppose $|S| = n$, that is, $S= \{s_1, \hdots, s_n\}$. Then each member has probability, that is, $p_i = P(\{s_i\})$ such that
$$p_i \geq 0 \text{ and } \sum_{i=1}^n p_i = 1$$ 

\section{Classical Equal Probability and Combinatorics}
\definition[permutation]
When there are $n$ elements, the number of events pulling $k$ elements out of $n$ elements is called a \tb{permutation} of $n$ elements taken $k$ at a time and denoted by $P_{n,k}$.
\theorem
$$P_{n,k} = n(n-1)\hdots(n-k+1) = \frac{n!}{(n-k)!}$$

\definition[combination]
The number of combinations of $n$ elements taken $k$ at a time is denoted by $C_{n,k}$ or $n \choose k$.
\theorem
$$C_{n,k} =  {n \choose k } = \frac{n!}{k!(n-k)!} = P_{n,k} / k!$$
\theorem[Binomial coefficients]
$$(x+y)^n = \sum_{k=0}^n {n \choose k} x^ky^{n-k}$$
\theorem[Newton Expansion]
For $|z| < 1$, the term $(1 + z)^r$ can be expanded as
$$(1+z)^r = \sum_{k=0}^\infty {r \choose k} z^k$$
\theorem
$${n \choose k} = \frac{r(r-1)\hdots(r-k+1)}{k!} = \frac{\Gamma(r + 1)}{\Gamma(r - k + 1)\Gamma(k + 1)}$$
with $\Gamma(\alpha) = \int_0^\infty x^{\alpha - 1}e^{-x} \, dx$
\theorem
For any numbers $x_1, \hdots, x_k$ and non-negative integer $n$,
$$(x_1 + \hdots + x_k)^n = \sum {n \choose n_1, \hdots, n_k} x_1^{n_1} \hdots x_k^{n_k}$$
It is easy to see that
\begin{equation}
\begin{aligned}\left(\begin{array}{c}{n} \\ {n_{1}, \ldots, n_{k}}\end{array}\right) &=\left(\begin{array}{c}{n} \\ {n_{1}}\end{array}\right)\left(\begin{array}{c}{n_{2}+\cdots+n_{k}} \\ {n_{2}}\end{array}\right)\left(\begin{array}{c}{n_{3}+\cdots+n_{k}} \\ {n_{3}}\end{array}\right) \cdots\left(\begin{array}{c}{n_{k}} \\ {n_{k}}\end{array}\right) \\ &=\frac{n !}{n_{1} ! \cdots n_{k} !} \end{aligned}
\end{equation}

\theorem[Stirling's formula]
$$\lim_{n \rightarrow \infty} \left| \log(n!) - [\frac{1}{2}\log(2\pi) + (n + \frac{1}{2})\log(n) - n] \right | = 0$$

\section{Inclusion-Exclusion Formula}
For any $n$ events $A_1, \hdots, A_n$,
\begin{equation}
\begin{aligned} P\left(\cup_{i=1}^n A_{i}\right)=& \sum_{i=1}^{n} P\left(A_{i}\right)-\sum_{i<j} P\left(A_{i} \cap A_{j}\right)+\sum_{i<j<k} P\left(A_{i} \cap A_{j} \cap A_{k}\right)+\cdots \\ &+(-1)^{n-1} P\left(A_{1} \cap \cdots \cap A_{n}\right) \end{aligned}
\end{equation}

\section{Conditional Probability}
\definition[conditional probability]
When $P(B) > 0$, the \tb{conditional probability} of an event $A$ given $B$ is defined by $$P(A|B) = P(A\cap B) / P(B)$$

\theorem
If $P(B) > 0$, then $P(A \cap B) = P(A|B)P(B)$.
\theorem
Let $A_1, \hdots, A_n$ be events with $P(A_1 \cap \hdots \cap A_n) > 0$. Then
\begin{equation}
P\left(A_{1} \cap \cdots \cap A_{n}\right)=P\left(A_{1}\right) P\left(A_{2} | A_{1}\right) P\left(A_{3} | A_{1}, A_{2}\right) \cdots P\left(A_{n} | A_{1}, \ldots, A_{n-1}\right)
\end{equation}

\section{Independence}
\definition[independence]
Two events $A$ and $B$ are \tb{independent} if and only if 
$$P(A\cap B) = P(A)P(B)$$.
A collection of events $\{A_i\}_{i\in I}$ are said to be \tb{(mutually) independent} if
$$P(\cap_{i\in J} A_i) = \Pi_{i\in J} P(A_i)$$
for any $\emptyset \neq J \subset I$. \\
A collection of events $\{A_i\}_{i\in I}$ are said to be \tb{pair-wise independent} if
$$P(A_i \cap A_j) = P(A_i)P(A_j)$$
for $i \neq j \in I$.

\theorem Two events $A$ and $B$ are independent if and only if $A$ and $B^c$ are independent.

\definition[conditionally independence]
Two events $A$ and $B$ are \tb{conditionally independent} given $C$ if
$$P(A\cap B | C) = P(A|C)P(B|C)$$
\remark Conditional independence does not imply independence.

\section{Bayes Theorem}
\definition
A collection of sets $B_1, \hdots, B_k$ is called a \tb{partition} of $A$ if and only if $B_1, \hdots, B_k$ are disjoint and $A = \cup_{i=1}^k B_i$.
\theorem[Law of total probability]
Let events $B_1, \hdots, B_k$ be a partition of $S$ with $P(B_j) > 0$ for all $j = 1,\hdots, k$. For any event $A$,
$$P(A) = \sum_{j=1}^k P(B_j)P(A|B_j)$$
\theorem[Bayes' Theorem]
If $ 0 < P(A), P(B) < 1$, then
$$P(B|A) = \frac{P(A|B)P(B)}{P(A|B)P(B) + P(A|B^c)P(B^c)}$$

\section{Random Variables}
\definition
A real-valued function $X$ on the sample space $S$ is called a \tb{random variable} if the probability of $X$ is well-defined, that is, $\{ s \in S: X(s) \leq r\}$ is an event for each $r \in \real$.

\definition[Borel sets in $\real$]
The collection of all Borel sets $\mc{B}$ in $\real$ is the smallest collection satisfying the followings
\begin{enumerate}
	\item $(a,b] \in \mc{B}$ for any $a < b \in \real$
	\item (closure under complement) For any $B \in \mc{B}, B^c \in \mc{B}$
	\item (closure under countable union) For any $B_1, B_2, \hdots \in \mc{B}, \cup_{j=1}^\infty B_j \in \mc{B}$
\end{enumerate}
We call the collection $\mc{B}$ the \tb{Borel $\sigma$-field}

\definition[Probability of a random variable]
For any Borel set $B$ in $\real$, an event $X \in B$ is defined as $\{s \in S: X(s) \in B\}$ and often denoted by $\{ X \in B\}$ or $(X \in B)$. The corresponding probability is
$$P(X \in B) = P(\{ s \in S: X(s) \in B\})$$
\lemma 
If $|X(S)| < \infty$ and $(X = r)$ is an event for any $r \in X(S)$, then $X$ is a random variable.

\definition[distribution]
The \tb{distribution} of $X$ is the collection of all probabilities of all events induced by $X$, that is, $(B, P(X \in B))$. Two random variables $X$ and $Y$ are said to be \tb{identically distributed} if they have the same distribution.

\remark
\blue{
To show X and Y having the same distribution, we need to check for any event $B$ on $\real$, $P(X \in B) = P(Y \in B)$. Since all Borel sets on $\real$ are induced by intervals, it is enough to prove $$P(a < X \leq b) = P(a < Y \leq b)$$ for any $a < b \in \real$. Even $P(X \leq a)=P(Y \leq a)$ for any $a \in \real$ guarantees that $X$ and $Y$ are identically distributed.
}

\definition[discrete random variable]
A random variable $X$ is said to be \tb{discrete} if $P(X = x) = 0$ or $P(X = x) > 0$ and $P(X \in \chi_0) = 1$ where $\chi_0 = \{ x \in \real: P(X = x) > 0\}$

\definition[probability mass function]
The \tb{probability mass function} (pmf) of a discrete random variable $X$ is
$$pmf_X(x) = P(X=x)$$
for any possible value of $x \in X(S)$.

\theorem
Let $X$ be a discrete random variable. Then the set of $x$ having $P(X = x)$ is at most countable.

\theorem
Let $f$ be the pmf of a discrete random variable $X$. The set of possible values of $X$ is $X(S) = \{x_1, x_2, \hdots\}$. For $x \notin X(S) \geq 0$ and $\sum_{i=1}^\infty f(x_i) = 1$.

\theorem
Let $X(S) = \{x_1, x_2, \hdots\}$ be the set of possible values of a discrete random variable $X$. Then for any subset $A$ of $\real$,
$$P(X \in A) = \sum_{x \in A}P(\{x\}) = \sum_{x \in A}pmf_X(x)$$
\definition[absolutely continuity and probability density function]
A random variable $X$ is said to be \tb{absolutely continuous} if the probability of each interval $[a, b]$ is of the form
$$P(a < X \leq b) = \int_a^b f(x)\,dx$$
where $a < b\in \real$ and $f$ is a non-negative function on $\real$. Such function $f$ is called a \tb{probability density function} (pdf) of $X$.
\theorem
Let $X$ be a continuous random variable. Then
$$pdf_X(x) = \frac{d}{dx}P(X\leq x)$$

\subsection{Examples of Random Variables}
\definition[Bernoulli]
A random variable $X$ taking value 0 or 1 with $P(X=1) = p$ and $P(X = 0) = 1 -p$ for some $p \in [0,1]$ is called a \tb{Bernoulli} random variable with success probability $p$ and often denoted by $X \sim$ Bernoulli($p$).

\definition[discrete uniform]
Let $\chi$ be a non-empty finite set. A random variable $X$ taking values in $\chi$ with equal probability is called a uniform random variable on $\chi$ and denoted by $X \sim uniform(\chi)$. \\
The probability mass function of $X \sim uniform(\chi)$ is
$$pmf_X(x) = \begin{cases}
	\frac{1}{|\chi|} &\text{if $x \in \chi$} \\
	0 &\text{otherwise}
\end{cases} $$
\definition[binomial]
A random variable $X$ is called a \tb{binomial} random variable if it has the same distribution as $Z$ which is the number of success in $n$ independent trails with success probability $p$, and denoted by $X \sim$ binomial$(n,p)$.\\
The probability mass function of $X \sim$ binomial$(n, p)$ is
$$pmf_X(x) = \begin{cases}
	{n \choose x} p^x(1-p)^{n-x} &\text{if $n = 0,1, \hdots$} \\
	0 &\text{otherwise}
\end{cases} $$

\definition[continuous uniform]
A random variable $X$ defined on $(a,b)$ for finite real numbers $a < b$ satisfying $P(c < X \leq d) = \frac{d-c}{b-a}$ for any $c,d$ such that $a\leq c \leq d\leq b$ is called a \tb{uniform} random variable on $(a,b)$ which is denoted by $X \sim$ uniform$(a,b)$.
The probability mass function of $X \sim uniform(a,b)$ is
$$pmf_X(x) = \begin{cases}
	\frac{1}{b-a} &\text{if $a < x<b$} \\
	0 &\text{otherwise}
\end{cases} $$

\definition[geometric]
Consider an independent Bernoulli trial with success probability $p$. The number of trials until the first success is called a \tb{geometric} distribution with parameter $p$, denoted by geometric($p$). \\
The geometric random variable $X \sim geometric(p)$ has probability mass function as
$$pmf_X(n) = (1-p)^{n-1}p$$
for $n \in \mb{N}$.



\definition[negative binomial]
Consider an independent Bernoulli trial with success probability $p$. The number of trials until $k$-th success is called a \tb{negative binomial} distribution with parameter $k$ and $p$, denoted by neg-bin($k$, $p$).\\
The negative binomial random variable $X \sim neg-bin(k, p)$ has probability mass function as
$$pmf_X(n) = {n-1 \choose k-1}(1-p)^{n-k}p^k$$
for $n \in \mb{N}$ s.t. $n \geq k$.

\definition[hypergeometric]
Consider a jar containing \(n\) balls of which \(r\) are black and the remainder \(n-r\) are white.
The random variable \(X\) is the number of black balls when \(m\) balls are drawn without
replacement. The probability of \(k\) black balls are drawn is
$$
\operatorname{pmf}_{X}(k)=\left\{\begin{array}{cc}{\left(\begin{array}{c}{n-r} \\ {m-k}\end{array}\right) /\left(\begin{array}{c}{n} \\ {m}\end{array}\right)} & {\text { if } k=0, \ldots, \min (r, m)} \\ {0} & {\text { otherwise. }}\end{array}\right.
$$
Such distribution is called a \tb{hypergeometric} distribution.

\definition[zeta/zipf]
A positive integer valued random variable \(X\) follows a \tb{Zeta} or \tb{Zipf} distribution if
$$
\operatorname{pmf}_{X}(n)=\frac{n^{-s}}{\zeta(s)}
$$
for \(n=1,2, \ldots\) and \(s>1\) where \(\zeta(s)=\sum_{n=1}^{\infty} n^{-s}\)

\definition[Poisson]
A \tb{Poisson} distribution with parameter $\mu > 0$ has the probability mass function
$$ pmf_X(n) = e^{-\mu}\frac{\mu^n}{n!}$$
for non-negative integer $n$.

\theorem
If $X \sim Poisson(\lambda)$ and the distribution of $Y$, conditional on $X = k$, is a binomial distribution, $Y | (X=k) \sim Binom(k,p)$, then the distribution of $Y$ follows a Poisson distribution $Y \sim Poisson(\lambda \cdot p)$

\theorem[Sums of Poisson-distributed random variables]
If $X_i \sim Poisson(\lambda_i)$ for $i = 1,\hdots,n$ are independent, and $\lambda = \sum_{i=1}^n \lambda_i$, then $Y = \left( \sum_{i=1}^n X_i\right) \sim Poisson(\lambda)$.


\definition[Exponential]
A continuous random variable \(W\) having the probability density
$$
\operatorname{pdf}_{W}(w)=\lambda e^{-\lambda w} 1(w>0)
$$
is distributed from an exponential distribution with parameter \(\lambda>0,\) which is denoted by
\(W \sim\) exponential \((\lambda)\).

\subsection{Cumulative Distribution Function}
The \tb{(cumulative) distribution function} of a random variable \(X\) is the function
$$
\operatorname{cdf}_{X}(x)=F_{X}(x)=P(X \leq x)
$$
for \(-\infty<x<\infty\).

\theorem[properties of distribution functions]
Let \(F\) be a distribution function. Then\\
(a) \(F\) is nondecreasing,\\
(b) \(\lim _{x \rightarrow \infty} F(x)=1\) and \(\lim _{x \rightarrow-\infty} F(x)=0\),\\
(c) \(F\) is right continuous, that is, \(\lim _{y} \searrow F(y)=F(x),\)\\
(d) \(F(x-):=\lim _{y}, \gamma F(y)=P(X<x)\)\\
(e) \(P(X=x)=F(x)-F(x-)\)

\theorem
If a real function $F$ satisfies (a)-(c) in the above properties, then it is a distribution function of a random variable.

\definition[$p$-quantile]
The $p$-quantile of a random variable $X$ is $x$ such that $P(X \leq x) \geq p$ and 
$P(X \geq x) \geq p$ and $P(X \geq x) \geq 1-p$.

\definition
 The median, lower quartile, upper quartile are 0.5-, 0.25-, 0.75-quantile. The inter quartile range (IQR) is the difference between upper and lower quartile.

\subsection{Multivariate Distributions}
\subsubsection{Bivariage Distributions}
\definition
The \tb{joint/bivariate distribution} of two random variables $X$ and $Y$ is the collection of all possible probabilities, that is, $P((X, Y) \in B)$ where $B$ is a Borel set in $\real^2$.

\definition
Two random variables $X$ and $Y$ are jointly continuously distributed if and only if there exists a non-negative function $f$ such that for any Borel set $B$ in $\real^2$
$$P((X, Y) \in B) = \iint_B f(x,y)\,dx\,dy$$
Such function $f$ is called a \tb{joint density function} of $(X, Y)$.

\theorem[Properties of joint density functions]
Joint density functions satisfies
\begin{enumerate}
	\item $$pdf_{X,Y}(x,y) \geq 0$$
	\item $$\iint pdf_{X, Y} (x,y) \,dx\,dy = 1$$
\end{enumerate}

\definition
The \tb{joint (cumulative) distribution function} of $X$ and $Y$ is
$$cdf_{X,Y}(x,y) = P(X \leq x, Y \leq y)$$


\definition
When $X$ and $Y$ are discrete, then the \tb{joint probability mass function} of $X$ and $Y$ is defined by
$$pmf_{X,Y}(x,y) = P(X=x, Y=y)$$



\theorem[Properties of joint probability mass functions] Satisfies \\
\begin{enumerate}
	\item $$pmf_{X,Y}(x,y) \geq 0$$
	\item $$\sum_{x,y} pmf_{X, Y} (x,y) = 1$$
\end{enumerate}

\theorem
Consider two random variables $X$ and $Y$.
\begin{align*}
	\underset{y \rightarrow - \infty}{\lim}cdf_{X,Y}(x,y) &= 0 \\
	\underset{x \rightarrow - \infty}{\lim}cdf_{X,Y}(x,y) &= 0 \\
	\underset{y \rightarrow  \infty}{\lim}cdf_{X,Y}(x,y) &= cdf_X(x) \\
	\underset{x \rightarrow  \infty}{\lim}cdf_{X,Y}(x,y) &= cdf_Y(y) \\
\end{align*}

\subsubsection{Marginal Distributions}
Suppose $X$ and $Y$ are random variables. The cdf or pmf or pdf of $X$ (or $Y$) derived from the joint cdf or pmf or pdf is called the \tb{marginal} cdf or pmf or pdf of $X$ (or $Y$).

\theorem
\begin{enumerate}
	\item $$pmf_X(x) = \sum_y pmf_{X,Y}(x,y)$$
	\item $$pdf_X(x) = \int pdf_{X,Y}(x,y)\, dy$$
\end{enumerate}

\definition
Two random variables $X$ and $Y$ are \tb{independent} if and only if 
$$P(X \in A, Y \in B) = P(X \in A)P(Y \in B)$$

\theorem
If two random variables $X$ and $Y$ are independent, then the following hold if the functions exist.
\begin{enumerate}
	\item $cdf_{X,Y}(x,y) = cdf_X(x) \times cdf_Y(y)$ for all $x,y$
	\item $pmf_{X,Y}(x,y) = pmf_X(x) \times pmf_Y(y)$ for all $x,y$
	\item $pdf_{X,Y}(x,y) = pdf_X(x) \times pdf_Y(y)$ for all $x,y$
\end{enumerate}

\theorem
If one of the following hold, then two random variables $X$ and $Y$ are independent.
\begin{enumerate}
	\item $cdf_{X,Y}(x,y) = cdf_X(x) \times cdf_Y(y)$ for all $x,y$
	\item $pmf_{X,Y}(x,y) = pmf_X(x) \times pmf_Y(y)$ for all $x,y$
	\item $pdf_{X,Y}(x,y) = pdf_X(x) \times pdf_Y(y)$ for all $x,y$
\end{enumerate}

\subsubsection{Conditional Distributions}
\definition
The conditional density of $X$ given $Y=y$ is
$$pdf_{X|Y}(x|y) = \frac{pdf_{X,Y}(x,y)}{pdf_Y(y)}$$

\theorem
$$pdf_{X,Y}(x,y) = pdf_X(x)pdf_{X|Y}(x|y)$$
\subsubsection{Multivariate Distributions}

\definition
The joint cumulative distribution function of $n$ variables $X_1, \hdots, X_n$ is defined by
$$cdf_{X_1, \hdots, X_n}(x_1, \hdots, x_n) = P(X_1 \leq x_1, \hdots, X_n \leq x_n)$$
The joint probability mass/density function of $n$ discrete/continuous random variables $X_1, \hdots, X_n$ is 
$$pmf_{X_1,\hdots,X_n}(x_1,\hdots,x_n) = P(X_1 = x_1, \hdots, X_n = x_n)$$
$$P((X_1, \hdots, X_n) \in B) = \int \underset{B}{\hdots}\int pdf_{X_1, \hdots, X_n}(x_1, \hdots, x_n) \,dx_n \hdots dx_1$$
\definition
Let $X_1, \hdots, X_n$ be random variables. Marginal cumulative distribution, probability mass, probability density functions of $X_1, \hdots, X_{i-1}, X_{i+1}, \hdots, X_n$ are
\begin{align}
	cdf_{X_1,\hdots,X_{i-1}, X_{i+1}, \hdots, X_n(x_1, \hdots, x_{i-1}, x_{i+1},\hdots, x_n)} = \underset{x_i \rightarrow \infty}{\lim}cdf_{X_1,\hdots,X_{i-1}, X_i, X_{i+1}, \hdots, X_n}(x_1, \hdots, x_{i-1}, x_i, x_{i+1},\hdots, x_n)\\
	pmf_{X_1,\hdots,X_{i-1}, X_{i+1}, \hdots, X_n(x_1, \hdots, x_{i-1}, x_{i+1},\hdots, x_n)} = \sum_{x_i}pmf_{X_1,\hdots,X_{i-1}, X_i, X_{i+1}, \hdots, X_n}(x_1, \hdots, x_{i-1}, x_i, x_{i+1},\hdots, x_n)\\
	pdf_{X_1,\hdots,X_{i-1}, X_{i+1}, \hdots, X_n(x_1, \hdots, x_{i-1}, x_{i+1},\hdots, x_n)} = \int pdf_{X_1,\hdots,X_{i-1}, X_i, X_{i+1}, \hdots, X_n}(x_1, \hdots, x_{i-1}, x_i, x_{i+1},\hdots, x_n) \, dx_i
\end{align}
\theorem
Let $X_1, \hdots, X_n$ be continuous random variables having cdf. Then
$$pdf_{X_1,\hdots,X_n}(x_1, \hdots, x_n) = \frac{\pd^n}{\pd x_1 \hdots \pd x_n}F(x_1,\hdots, x_n)$$
\definition
Random variables $X_1, \hdots, X_n$ are \tb{independent} if and only if for any Borel sets $B_1, \hdots, B_n$
$$P(X_1 \in B_1, \hdots, X_n \in B_n) = P(X_1 \in B_1)\hdots P(X_n \in B_n)$$
\theorem
Random variables $X_1, \hdots, X_n$ are \tb{independent} if and only if 
$$cdf_{X_1, \hdots, X_n}(x_1,\hdots,x_n) = cdf_{X_1}(x_1)\hdots cdf_{X_n}(x_n)$$
\subsection{Functions of Random Variables}
\theorem
Let X be a discrete random variable and $Y = g(X)$ be a \under{transformed random variable} where $g: \real \rightarrow \real$ is a function. The pmf of $Y$ is
$$pmf_Y(y) = \sum_{x:g(x) = y}pmf_X(x)$$
\theorem
Let X be a continuous random variable and $Y = g(X)$ be a \under{transformed random variable} where $g$ is an appropriate transformation like continuous increasing. The cdf of $Y$ is
$$cdf_Y(y) = \int_{\{x:g(x) \leq y\}} pdf_X(x) \, dx$$
The probability density function of $Y$ is
$$pdf_Y(y) = \frac{d}{dy} cdf_Y(y)$$

\theorem 
Let $X$ be a continuous random variable and $F(x) = cdf_X(x)$. Then new random variable $Y=F(X)$ is uniformly distributed on $(0,1)$, that is, $Y \sim uniform(0,1)$.

\theorem[change of variable]
Let $X$ be a continuous random variable and $g$ be a one-to-one and differentiable function. Then the density of random variable $Y = g(X)$ is
$$pdf_Y(y) = pdf_X(g^{-1}(y))\abs{\frac{d}{dy}g^{-1}(y)}$$
whenever $y$ is in the range of $Y(S)$.

\theorem
Consider discrete random variables $X_1, \hdots, X_n$. There exist $m$ functions $g_1, \hdots, g_m$ so that $Y_i = g_i(X_1, \hdots, X_n)$. The joint probability mass function of $Y = (Y_1, \hdots, Y_m)$ is 
$$pmf_Y(y) = \sum_{x:g_i(x) = y_i, i=1,\hdots,m} pmf_X(x)$$
\definition
Random variables $X_1, \hdots, X_n$ are said to be \tb{independent} and \tb{identically distributed (i.i.d)} if all random variables have the same distribution and are independent.

\theorem
Let $X$ and $Y$ be jointly continuous random variables. The density of $Z = X + Y$ is
$$pdf_Z(z) = \int pdf_{X,Y}(x,z-x) \, dx$$
If $X$ and $Y$ are independent, then
$$pdf_X(z) = \int pdf_X(x)pdf_Y(z-x)\, dx$$

\theorem[change of variable]
Suppose $X_1, \hdots, X_n$ have a joint density function $f(x_1, \hdots, x_n)$ and $Y_i = g_i(X_1, \hdots, X_n)$ for one-to-one correspondent and differentiable functions $g_i$'s, say $y = g(x)$. The joint density of $Y_1, \hdots, Y_n$ is
$$pdf_Y(y) = pdf_X(x) \abs{ \det \left ( \frac{\pd(x_1, \hdots, x_n)}{\pd(y_1, \hdots, y_n)} \right)}$$
where $x = (x_1, \hdots, x_n) = g^{-1}(y)$

\subsection{Expectation}
\definition{expectation}
The \tb{expectation} (or expected value or mean value) of a discrete random variable is 
$$\expect{X} = \sum_x x \times P(X=x) = \sum_x x \times pmf_X(x)$$
when the sum is absolutely convergent.
\definition
The expectation of a continuous random variable $X$ is defined by
$$\expect{X} = \int x \times pdf_X(x) \, dx$$

\theorem
Assume a discrete random variable $X$ is non-negative. Then
$$\expect{X} = \int_0^\infty P(X > z)\, dz = \int_0^\infty x \, dF(x)$$

\corollary
Let $X$ be a non-negative integer valued random variables. Then
$$\expect{X} = \sum_{n=1}^\infty P(X \geq n)$$

\lemma
Let $F$ be the cumulative distribution function of a random variable $X$. For an interval, \red{$$P(a < X \leq b) = \expect{1(a < X \leq b)}$$} In general, for each event $A$ of $X$,
\red{$$P(X \in A) = \expect{1 (X \in A)}$$}

\theorem
For any random variable $X$ with finite expectation,
$$\expect{X} = \int_0^\infty P(X > z)\,dz - \int_{-\infty}^0 P(X < z) \, dz = \int_{-\infty}^\infty x \, dF(x)$$

\theorem
Let $X$ be a random variable and $g$ be a function on $\real$. If expectation of $Y = g(X)$ is defined, then
$$\expect{Y} = \int g(x)\, d\, cdf_X(x) = \int_{-\infty}^\infty g(x) \cdot pdf_X(x) \, dx$$ or 
$$\expect{Y} = \int g(x)\, d\, cdf_X(x) = \sum_x g(x) \cdot pdf_X(x)$$

\lemma
Assume $X, Y \geq 0$ with probability 1, that is, $P(X \geq 0, Y \geq 0) = 1$, then
$$\expect{X + Y} = \expect{X} + \expect{Y}$$
and
$$\expect{X - Y} = \expect{X} - \expect{Y}$$

\theorem[Properties of Expectation]
Satisfies\\
\begin{enumerate}
	\item (linearity) Let $Y = aX + b$, then $$\expect{Y} = a\expect{X} + b$$
	\item (monotonicity) If $X \geq 0$, that is, $P(X \geq 0) = 1$, then $E(X) \geq 0$
	\item (additivity) $\expect(X + Y) = \expect{X} + \expect{Y}$
	\item For constant random variable 1, $\expect{1} = 1$
\end{enumerate}

\theorem
Let $X$ and $Y$ be two independent random variables and $g$ and $h$ be real functions satisfying $g(X)$ and $h(Y)$ are random variables with finite expectations. Then
$$\expect{g(X)h(Y)} = \expect{g(X)}\expect{h(Y)}$$

\subsection{Moments}
\definition
For positive integer $k$, the \tb{$k$-th moment} of $X$ is $\expect{X^k}$ and the \tb{$k$-th  central moment} is $\expect{(X - \expect{X})^k}$.

\theorem
If $\expect{|X|^t} < \infty$ for some $t > 0$, then $\expect{|X|^s} < \infty$ for any $0 \leq s \leq t$.

\definition[variance]
The \tb{variance} of a random variable $X$ is
$$\var{X} = \expect{(X - \expect{X})^2}$$
The \tb{covariance} and \tb{correlation} between two random variables $X$ and $Y$ are
$$Cov(X,Y) = \expect{(X - \expect{X})(Y - \expect{Y})}$$ and
$$Cor(X,Y) = \frac{Cov(X, Y)}{\sqrt{\var{X}\var{Y}}}$$

\theorem[Properties of variance]
satisfies
\begin{enumerate}
	\item $\var{X} \geq 0$
	\item $\var{X} = \expect{X^2} - (\expect{X})^2$
	\item $\var{aX + b} = a^2 \var{X}$
	\item $\var{X + Y} = \var{X} + \var{Y} + 2Cov(X, Y)$
	\item $\var{X + Y} = \var{X} + \var{Y}$ if and only if $X$ and $Y$ are uncorrelated.
	\item \blue{If a random variable $X$ is bounded, then it must has finite variance.}
	\item $\var{X} = 0$ if and only if $P(X = c) = 1$ for some $c \in \real$.
\end{enumerate}
\theorem[Properties of covariance]
$$Cov[X, Y] = \expect{X, Y} - \expect{X}\expect{Y}$$


\definition[skewness and kurtosis]
The standardized third and fourth moments are said to be \tb{skewness} and \tb{kurtosis}, that is, \\
skewness $= \expect{(X - \mu)^3} / \sigma^3$, \quad kurtosis $= \expect{(X - \mu)^4} / \sigma^4$ \\
where $\mu = \expect{X}$ and $\sigma^2 = \var{X}$.


\section{Inequalities}
\theorem[Chebychev's inequality]
Let $X$ be a random variable with mean $\mu$ and variance $\sigma^2$. Then, for any $\alpha > 0$,
$$P(|X - \mu| \geq \alpha\sigma) \leq \frac{1}{\alpha^2}$$
Equivalently, for $\alpha > 0$,
$$\red{P(|X - \mu| > \alpha) \leq \frac{\var{X}}{\alpha^2}}$$

\theorem[Markov's inequality]
If $X \geq 0$ with $\mu = \expect{X} < \infty$, then for any $\alpha > 0$, 
$$P(X \geq \alpha) \leq \mu / \alpha$$

\remark
The Chebychev's inequality is a special case of Markov's inequality by considering $$Y = (X - \mu)^2$$
Note that
\(A=\{s \in \Omega:| X(s)-E(X) | \geq r\}=\left\{s \in \Omega :(X(s)-E(X))^{2} \geq r^{2}\right\}\) \\
Now, consider the random variable, \(Y,\) where
\(Y(s)=(X(s)-E(X))^{2} .\) \\
Note that \(Y\) is a non-negative random variable. \\
Thus, we can apply Markov's inequality to it, to get: \\
\(P(A)=P\left(Y \geq r^{2}\right) \leq \frac{E(Y)}{r^{2}}=\frac{E\left((X-E(X))^{2}\right)}{r^{2}}=\frac{V(X)}{r^{2}} . \)
\theorem[Cauchy-Schwartz' inequality]
Let $X$ and $Y$ be two random variables having finite second moment. Then
$$[\expect{XY}]^2 \leq \expect{X^2}\expect{Y^2}$$
where the equality holds if and only if $P(aX = bY) = 1$ for some $a, b \in \real$.

\theorem
Let $X$ and $Y$ be two random variables with finite second moment. Then $Y = aX + b$ for some $a,b$ if and only if $|Corr(X,Y)| = 1$.

\lemma[Young's inequality]
For $p,q > 1$ with $1/p + 1/q = 1$ and two nonnegative real numbers $x,y \geq 0$,
$$xy \leq x^p/p + y^q /q$$

\theorem[H\"older's inequality]
For $p,q > 1$ with $1/p + 1/q = 1$, $$\expect{|XY|} \leq ||X||_p ||Y||_q$$ when the expectations exist and are finite where
$||X||_r = \expect{|X|^r}^{1/r}$ for $r > 0$.

\remark The Cauchy-Schwartz' inequality is a special case of H\"older's inequality ($p=q=2$)

\theorem[Jensen's inequality]
For a convex function $\varphi$, 
$$\varphi(\expect{X}) \leq \expect{\varphi(X)}$$

\theorem[Minkowski's inequality]
For $p \geq 1$, 
$$||X + Y||_p \leq ||X||_p + ||Y||_p$$

\section{Conditional Expectation}
\definition{conditional expectation}
The conditional expectation of $Y$ given $X = x$ is defined by
$$\expect{Y | X=x} = \int y \, dcdf_{Y|X}(y|x)$$

\remark The conditional expectation $\expect{Y|X=x}$ is always a function of $x$, say $h(x)$. Then denote $h(X) = \expect{Y|X}$ as a random variable.

\theorem
Assume $\expect{|Y|} < \infty$. Then
$$\expect{Y|X=x} = \int_0^\infty P(Y > z|X=x) \, dz - \int_{-\infty}^0 P(Y<z|X=x)\, dz$$
If $Y$ is discrete, then
$$\expect{Y|X=x} = \sum_y y \times pmf_{Y|X}(y|x)$$
If $Y$ is continuous, then
$$\expect{Y|X=x} = \int y \times pmf_{Y|X}(y|x) \, dy$$

\theorem[Properties of conditional expectation]
Satisfies
\begin{enumerate}
	\item $\expect{aY + b| X} = a\expect{Y|X} + b$
	\item If $P(Y \geq 0|X) = 1$, then $\expect{Y|X} \geq 0$
	\item $\expect{Y + Z |X} = \expect{Y|X} + \expect{Z|X}$
	\item for constant random variable 1, $\expect{1|X} = 1$
	\item for convex function $\phi$, $\expect{\varphi(Y)|X} \geq \varphi(\expect{Y|X})$
\end{enumerate}

\theorem[Law of Total Expectation]
$$\expect{\expect{Y|X}} = \expect{Y}$$
i.e. The expected value of the conditional expected value of $Y$ given $X$ is the same as the expected value of $Y$. \\
One special case states that if $\{A_i\}_i$ is a finite or countable partition of the sample space, then
$$\expect{X} = \sum_i \expect{X|A_i}P(A_i)$$

\definition{conditional variance}
The conditional variance is given by
$$\var{Y|X=x} = \expect{(Y - \expect{Y|X=x})^2|X=x}$$

\theorem
$$\var{Y} = \expect{\var{Y|X}} + \var{\expect{Y|X}}$$

\section{Probability Related Functions}
Let $X$ be a random variable. 
\begin{enumerate}
	\item  \tb{moment generating function}: $mgf_X(t) = \expect{e^{tX}}$
	\item  \tb{cumulant generating function}: $cgf_X(t) = \log\expect{e^{tX}}$
	\item  \tb{probability generating function}: $pgf_X(t) = \expect{z^{X}}$
    \item  \tb{characteristic generating function}: $chf_X(t) = \expect{e^{itX}}$ \\
\end{enumerate}
where $t \in \real, z > 0$ and $i = \sqrt{-1}$ is the unit imaginary number.

\theorem[properties of mgf]
As follows
\begin{enumerate}
	\item $mgf_X(0) = 1$
	\item $\expect{X^k} = \frac{d^k}{dt_k} mgf_X(0)$ if it exists
	\item If $\expect{|X|^k} < \infty$, then for $\mu_j = \expect{X^j}$ where $j = 1, \hdots, k$, 
	$$mgf_X(t) = 1 + \mu_1t+ \mu_2\frac{t^2}{2!} + \hdots + \mu_k \frac{t^k}{k!} + o(|t|^k)$$
	\item $mgf_{aX+b}(t) = e^{bt}mgf_X(at)$
	\item If $X$ and $Y$ are independent, then $$mgf_{X,Y}(s,t) = mgf_X(s)mgf_Y(t)$$
\end{enumerate}

\theorem[properties of cgf]
As follows
\begin{enumerate}
	\item $cgf_X(0) = 0$
	\item If $X$ and $Y$ are independent, then $$cgf_{X,Y}(s,t) = cgf_X(s) + cgf_Y(t)$$
\end{enumerate}

\theorem[properties of pgf]
As follows
\begin{enumerate}
	\item $pgf_X(1) = 1$
	\item $\expect{X(X-1)\hdots(X-k+1)} = \frac{d^k}{dz^k}pgf_X(1)$ if it exists.
	\item  If $X$ and $Y$ are independent, then $$pgf_{X,Y}(s,t) = pgf_X(s) + pgf_Y(t)$$
\end{enumerate}
	
\theorem[properties of chf]
As follows
\begin{enumerate}
	\item $chf_X(0) = 1$
	\item $\expect{X^k} = (i)^{-k}\frac{d^k}{dt^k} chf_X(0)$ if it exists
	\item If $\expect{|X|^k} < \infty$, then for $\mu_j = \expect{X^j}$ where $j = 1, \hdots, k$, 
	$$chf_X(t) = 1 + i\mu_1t - \mu_2\frac{t^2}{2!} + \hdots + i^k \mu_k\frac{t^k}{k!} + o(|t|^k)$$
	\item $chf_{aX+b} = e^{ibt}chf_X(at)$
	\item If $X$ and $Y$ are independent, then $$chf_{X,Y}(s,t) = chf_X(s)chf_Y(t)$$
	\item $|chf_X(t)| \leq 1$ for all $t$
	\item chf is uniformly continuous
	\item for any $t_1, \hdots, t_n \in \real$ and $z_1, \hdots, z_n \in \mb{C}$, $$\sum_{j,k}chf_X(t_j - t_k)z_j\bar{z}_k \geq 0$$
\end{enumerate}

\theorem If two random variables $X$ and $Y$ have the same moment generating functions in an open neighbourhood of 0, that is, $(-a, b)$ for $a,b > 0$, then $X$ and $Y$ are identically distributed.

\theorem
If a function $\varphi: \real \rightarrow \mb{C}$ satisfies 5 - 8 in Theorem 11.4, then there exists a random variable having $\varphi$ as its characteristic function.

\definition
The joint probability/moment/cumulant generating and characteristic functions of $X$ and $Y$ are
\begin{enumerate}
	\item $mgf_{X, Y}(s,t) = \expect{e^{sX + tY}}$
	\item $cgf_{X, Y}(s,t) = \log mgf_{X,Y}(s,t)$
	\item $pgf_{X, Y}(s,t) = \expect{s^Xt^Y}$
	\item $chf_{X, Y}(s,t) = \expect{e^{isX + itY}}$
\end{enumerate}

\theorem[Inversion Formula]
Let $\varphi$ be a characteristic function of a random variable $X$. Then for any $a, b$,
$$P(a < X < b) + \{P(X=a) + P(X=b)\}/2 = \underset{T \rightarrow \infty}{\lim} \frac{1}{2\pi} \int_{-\infty}^\infty \frac{e^{-iat} - e^{-ibt}}{it} \varphi(t) \, dt$$

\theorem[\red{Chernoff Bound}]
Let $X$ be a random variable having moment generating function. For any constant $x$,
$$P(X \geq x) \leq \underset{t > 0}{\inf}\,e^{-xt}mgf_X(t)$$

\subsection{Survival Functions}
Let $X$ be a non-negative valued random variable. \\
The \tb{survival} function of $X$ is $S_X(t) = P(X > t)$ or $S_X(t) = 1 - F_X(t)$. \\
(the probability of surviving longer than time $x$.\\
The \tb{hazard} function is
$$h_X(t) = \frac{pdf_X(t)}{S_X(t)} = \frac{pdf_X(t)}{1-F_X(t)}$$
(measures the risk of event (or death) at time $x$.
The \tb{cumulative hazard} function is
$$H_X(t) = \int_0^t h_X(z) \, dz$$ for $t > 0$. \\
The \tb{residual} (or future) lifetime given $X > t$ is defined by
$$R_X(t) = X - t$$
The \tb{mean residual lifetime} is the conditional expectation of residual lifetime given $X > t$, that is,
\begin{align}
	\expect{R_X(t) | X > t} = \int_0^\infty P(R_X(t) > z |X > t)\, dz = \int_t^\infty \frac{S_X(z)}{S_X(t)}
\end{align}
Particularly for $t = 0$ and $S_X(0) = 1$,
$$\expect{R_X(0)|X>0} = \int_0^\infty S_X(z) \, dz = \expect{X}$$

\section{Stochastic process}
\definition
A \tb{stochastic process} os a collection of time indexed random variables $$\{X_t: t \in \mc{T}\}$$
A collection of $\sigma$-field $\mc{F} = \{\mc{F}_t\}_{t \in \mc{T}}$ is called a \tb{filtration} if $\mc{F} \subset \mc{F}_t$ for any $0 \leq s \leq t$. \\
A stochastic process $X = \{X_t\}_{t \in \mc{T}}$ is said to be \tb{adapted to the filtration $\mc{F}$} if $X_t$ is $\mc{F}_t$-measurable (or $\{X_t \leq r\} \in \mc{F}_t$ for any real number $r$).

\definition[Martingales]
A stochastic process $X_n$ is said to be a (discrete-time) \tb{martingale} if
\begin{enumerate}
	\item $\expect{|X_n|} < \infty$
	\item $\expect{X_{n+1} | X_0, \hdots, X_n} = X_n$ for all $n$
	\item A stochastic process $X_n$ is said to be \tb{supermartingale} if it satisfies above (1) and
	$$\expect{X_{n+1} | X_0, \hdots, X_n} \leq X_n$$ for all $n$.
	\item A stochastic process $X_n$ is said to be \tb{submartingale} if it satisfies above (1) and
	$$\expect{X_{n+1} | X_0, \hdots, X_n} \geq X_n$$ for all $n$.
\end{enumerate}
Note: the condition $X_0, \hdots, X_n$ is often replaced by $\mc{F}$, that is,
$$\expect{X_{n+1}|\mc{F}_n} = X_n$$
\remark A martingale is both supermartingale and submartingale. \\
If $X_n$ is a submartingale, then $-X_n$ is a supermaringale.

\definition[stopping time]
A time valued random variable $T$ is said to be a \tb{stopping time} if the event $\{T \leq n\}$ can be expressed by $X_0, \hdots, X_n$
\example
The first time $T$ that the stochastic process $X_n$ is bigger than or equal to a constant $K$ is a stopping time by considering
$$\{T = n\} = \{X_1 < K, \hdots, X_{n-1} < K, X_n \geq K\}$$
\theorem[Optional Sampling Theorem]
Let $X_n$ be a submartingale and T is a stopping time with $P(T \leq k) = 1$. Then
$$\expect{X_0} \leq \expect{X_T} \leq \expect{X_k}$$

\subsection{Random Walk}
Let $X_1, X_2, \hdots$ be a sequence of independent random variables having mean zero and variance 1. Define $S_n = X_1 + \hdots + X_n$
\theorem
For any $\alpha > 0$,
$$P(\underset{k = 1,\hdots, n}{\max} |S_k| \geq \alpha) \leq \frac{\var{S_n}}{\alpha^2}$$

\theorem
If $X_n$ is symmetric for each $n$, then
$$P(\underset{k = 1,\hdots, n}{\max} |S_k| \geq \alpha) \leq 2P(S_n \geq \alpha)$$

\subsection{Poisson Process}
A \tb{Poisson process with intensity $\lambda$} is a stochastic process $N = \{N_t: t \geq 0\}$ taking values in non-negative integers satisfying \\
(a) $N_0 = 0$ and $N_s \leq N_t$ if $0 \leq s \leq t$ \\
(b) $P(N_{t+h} = n+m |N_t = n) = \begin{cases}
	1 - \lambda h + o(h) &\text{if $m = 0$} \\
	\lambda h + o(h) &\text{if $m = 1$} \\
	o(h) &\text{if $m > 1$}
\end{cases}$ \\
(c) For $0 \leq s < t$, the arrivals $N_t - N_s$ in the interval $(s, t]$ is independent of the arrivals $N_s$ in the interval $(0, s]$.

\theorem
For any fixed time $t > 0$, $N_t \sim Poisson(\lambda t)$
\theorem
The interarrival times $X_1, X_2, \hdots$ are independent and identically distributed from exponential with $\lambda$

\subsection{Reflection principle (Wiener process)}
\definition[Wiener Process]
A continuous-time stochastic process $W(t)$ for $t \geq 0$ with $W(0) = 0$ and such that the increment $W(t) - W(s)$ is Gaussian with mean $0$ and variance $t - s$ for any $0 \leq s < t$, and increments for nonoverlapping time intervals are independent. 
\remark
Brownian motion (i.e. random walk with random step sizes) is the most common example of a Wiener process.
\theorem[Reflection principle]
If $(W(t): t \geq 0$) is a Wiener process, and $a > 0$ is a threshold, then
$$P\left( \underset{0 \leq s\leq t}{\sup} W(s) \geq a\right) = 2 P(W(t) \geq a)$$

\remark
If the path of a Wiener process $f(t)$ reaches a value $f(s) = a $ at time $t = s$, then the subsequent path after time $s$ has the same distribution as the reflection of the subsequent path about the value $a$. 


\section{Mode of Convergence}
\definition
Modes of convergence
\begin{itemize}
	\item A sequence of random variables $X_n$ converges to $X$ \tb{in distribution} ($X_n \overset{d}{\longrightarrow} X$) if $$P(X_n \leq x) \rightarrow P(X \leq x)$$ as $n \rightarrow \infty$ for any $x$ with $P(X = x) = 0$.
	\item  A sequence of random variables $X_n$ converges to $X$ \tb{in probability} ($X_n \overset{p}{\longrightarrow} X$) if $$P(|X_n - X| > \epsilon) \rightarrow 0$$ as $n \rightarrow \infty$
	\item A sequence of random variables $X_n$ converges to $X$ \tb{almost surely} ($X_n \overset{a.s.}{\longrightarrow} X$) if $$P(\lim \sup_{n \rightarrow \infty} |X_n - X| = 0) = 1$$
	\item A sequence of random variables $X_n$ converges to $X$ \tb{in $L^p$} ($X_n \overset{L^p}{\longrightarrow}X$) for $p > 0$ if $$\expect{|X_n - X|^p} \rightarrow 0$$ as $n \rightarrow \infty$
\end{itemize}

\theorem
Let $X_n$ and $X$ be discrete random variables with probability mass functions $f_n(x)$ and $f(x)$ satisfying $f_n(x) \rightarrow f(x)$ for any $x$ with $f(x) > 0$. Then $$X_n \longrightarrow X$$ in distribution.

\theorem[Relations between modes of convergence] As follows:\\
(a) $X_n \overset{a.s.}{\longrightarrow} X \implies X_n \overset{p}{\longrightarrow} X$ \\
(b) $X_n \overset{L^p}{\longrightarrow} X \implies X_n \overset{p}{\longrightarrow} X$ \\
(c) $X_n \overset{p}{\longrightarrow} X \implies X_n \overset{d}{\longrightarrow} X$ \\

\subsection{$L^1$ Convergence}
\lemma[$L^1$ Convergence]
If $Y \geq 0$ and $\expect[Y] < \infty$, then for any $\epsilon > 0$ there exists $M > 0$ such that $$\expect{Y\id{Y > M}} < \epsilon$$

\lemma
Suppose a random variable $Y$ has a finite absolute expectation, that is, $\expect{|Y|} < \infty$. For any $\epsilon > 0$, there exists $\delta > 0$ such that $|\expect{Y\id{A}}| < \epsilon$ for any event $A$ with $P(A) < \delta$ where $\id{A}$ is an indicator function of the event $A$. 

\lemma
Suppose a random variable $Y$ has a finite absolute expectation, that is, $\expect{|Y|} < \infty$ and a sequence $A_n$ of events satisfy $P(A_n) \rightarrow 0$. Then $$\expect{Y\id{A_n}} \rightarrow 0$$

\theorem[Dominated Convergence Theorem]
Suppose that $X_n \rightarrow X$ in probability, $|X_n| \leq Y$ and $\expect{Y} < \infty$. Then
$$\expect{X_n} \rightarrow \expect{X}$$

\theorem[Generalized Dominated Convergence Theorem]
If all $X, Y, X_n, Y_n$ have finite absolute expectation, $|X_n| \leq Y_n$ for all $n$, $X_n \rightarrow X$ in probability, $Y_n \rightarrow Y$, and $\expect{Y_n} \rightarrow \expect{Y}$, then $$\expect{X_n} \rightarrow \expect{X}$$

\theorem[Monotone Convergence Theorem]
Let $X_n$ be non-negative non-decreasing random variables. Suppose $\underset{n \rightarrow \infty}{\lim} X_n = X$ is finite a.s. Then $$\underset{n \rightarrow \infty}{\lim} \expect{X_n} = \expect{X}$$ 
 
\theorem[Fatou's lemma]
Let $X_1, X_2, \hdots$ be a sequence of non-negative random variables. Then
$$\expect{\underset{n \rightarrow \infty}{\lim} \inf X_n} \leq \underset{n \rightarrow \infty}{\lim}{\inf} \,\expect{X_n}$$


\subsection{Almost Sure Convergence}
\theorem[Borel-Cantelli lemma]
Let $A = \cap_{m=1}^\infty\cup_{n=m}^\infty A_n$ be the event that infinitely many $A_n$'s occur.
\begin{enumerate}
	\item $P(A) = 0$ if $\sum_n P(A_n) < \infty$
	\item $P(A) = 1$ if $\sum_n P(A_n) = \infty$ and $A_1, A_2, \hdots$ are independent.
\end{enumerate}

\theorem
If for any $\epsilon > 0$, $\sum_{n=1}^\infty P(|X_n - X| > \epsilon) < \infty$, then $X_n \rightarrow X$ almost surely.

\theorem If a sequence of random variables $X_n$ converges to $X$ in probability, then there exists a subsequence $n_k$ such that $X_{n_k}$ converges to $X$ almost surely.

\theorem
A sequence $x_n$ of real numbers converges to $x$ if and only if for any subsequence $n_k$ there exists a further subsequence $n_{k_l}$ such that $x_{n_{k_l}}$

\theorem
A sequence of random variables $X_n$ converges to $X$ in probability if and only if for any subsequence $n_k$ there exists a further subsequence $n_{k_l}$ such that $X_{n_{k_l}}$ converges to $X$ a.s. 

\subsection{Convergence in distribution}
\theorem
As follows \\
(a) If $X_n \overset{d}{\longrightarrow} c$ where $c$ is a constant, then $X_n \overset{p}{\rightarrow} c$. \\
(b) If $X_n \overset{p}{\longrightarrow} c$ and $P(|X_n| \leq M) = 1$ for some $M > 0$, then $X_n \overset{L^p}{\longrightarrow} X$ for any $p > 0$

\theorem
Let $X$ be a random variable with $P(X=x) = 0$ for all $x$ and $F$ be the distribution function of $X$. Then $F(X) \sim uniform(0,1)$ and $F^{-1}(U) \sim X$ for any $U \sim uniform(0,1)$

\theorem[Skorokhod's representation theorem]
If $X_n \overset{d}{\longrightarrow} X$, then there exist random variables $Y, Y_1, Y_2, \hdots$ in a probability space such that \\
(a) $X_n$ and $Y_n$ have the same distribution as well as $X$ and $Y$ have the same distribution \\
(b) $Y_n \overset{a.s.}{\longrightarrow} Y$

\theorem[Continuous mapping theorem]
Let $g$ be a continuous function.
\begin{enumerate}
	\item $X_n \overset{a.s.}{\longrightarrow} X \implies g(X_n) \overset{a.s.}{\longrightarrow} g(X)$
	\item $X_n \overset{p}{\longrightarrow} X \implies g(X_n) \overset{p}{\longrightarrow} g(X)$
	\item $X_n \overset{d}{\longrightarrow} X \implies g(X_n) \overset{d}{\longrightarrow} g(X)$
\end{enumerate}

\theorem
$X_n \overset{d}{\longrightarrow} X$ if and only if $\expect{g(X_n)} \rightarrow \expect{g(X)}$ for any bounded continuous function $g$.

\theorem
$X_n \overset{d}{\longrightarrow} X$ if and only if $$chf_{X_n}(t) \rightarrow chf_X(t)$$

\theorem
If $X_n \overset{d}{\longrightarrow} X$, then
$$aX_n + b \overset{d}{\longrightarrow} aX + b$$ for any $a,b \in \real$

\theorem[Slutsky's lemma]
Suppose $X_n \overset{d}{\longrightarrow} X$ and $Y_n \overset{d}{\longrightarrow} c$ for a constant $c$.
\begin{enumerate}
	\item $X_n + Y_n  \overset{d}{\longrightarrow} X + c$
	\item $X_nY_n  \overset{d}{\longrightarrow} Xc$
	\item $X_n/Y_n \overset{d}{\longrightarrow} X/c$ if $c \neq 0$
\end{enumerate}

\section{Law of Large Numbers}

\theorem[Weak Law of Large Numbers]
Let $X_n$ be i.i.d. with $\expect{|X_n|} < \infty$. Then $\bar{X}_n \rightarrow \expect{X_1}$ \red{in probability.}

\theorem[Strong Law of Large Numbers]
Let $X_1, \hdots, X_n$ be i.i.d. r.v.s with $\expect{|X_n|} < \infty$. Then \blue{$$\bar{X}_n \overset{a.s.}{\longrightarrow} \expect{X_1}$$}

\theorem
Let $X_1, \hdots, X_n$ be i.i.d. r.v.s with $\expect{X_n^2} < \infty$. For $\mu = \expect{X_1}$. $$\bar{X}_n = (X_1 + \hdots + X_n) / n \rightarrow \mu$$ \red{almost surely and in $L^2$.}

\section{Central Limit Theorem}
For $k \approx np$, the binomial probability is approximated by
$${n \choose k}p^k(1-p)^{n-k} \approx \frac{1}{\sqrt{2\pi np(1-p)}}\exp{\left ( - \frac{(k - np)^2}{2np(1-p)}\right)}$$

\theorem[Levy's Central Limit Theorem]
Let $X_1, \hdots, X_n$ be i.i.d. r.v.s with $\mu = \expect{X_i}$ and $\sigma^2 = \var{X_i}$. Then
$$\sqrt{n}(\bar{X}_n - \mu) / \sigma \overset{d}{\longrightarrow} N(0,1)$$

\theorem[Lindeberg-Feller Central Limit Theorem]
Let $X_1, \hdots, X_n$ be i.i.d. r.v.s with $\expect{X_i} = 0$ and $\sigma_i^2 = \var{X_i^2} < \infty$. Let $s_n^2 = \expect{X_1^2} + \hdots + \expect{X_n^2}$ The Lindeberg condition
$$\frac{1}{s_n^2 \sum_{k=1}^n \expect{X_k^2\id{X_k^2 > \epsilon s_n^2}}} \rightarrow 0$$ for any $\epsilon > 0$ holds if and only if
$$(X_1 + \hdots + X_n) /s_n \overset{d}{\longrightarrow} N(0,1)$$ and $$\max(\sigma_1^2, \hdots, \sigma_n^2)/ s_n^2 \rightarrow 0$$

\theorem[Lyapounov's condition]
Let $X_1, \hdots, X_n$ be i.i.d. r.v.s with $\expect{X_i} = 0$ and $\sigma_i^2 = \var{X_i^2} < \infty$ satisfying Lyapounov's condition
$$\underset{n \rightarrow \infty}{\lim} \frac{1}{s_n^{2 + \delta}} \sum_{k=1}^n \expect{|X_k|^{2 + \delta}} = 0$$
Then Lindeberg's condition holds. Hence $$(X_1 + \hdots + X_n) /s_n \overset{d}{\longrightarrow} N(0,1)$$


\theorem[$\delta$-method]
Let $X_1, \hdots, X_n$ be i.i.d. r.v.s anf $a_n$ is a sequence of positive real numbers diverging to infinity. If $a_n(X_n - \mu) \overset{d}{\longrightarrow} Z$ for some r.v. Z and a constant $\mu$, then for any continuously differentiable function $g$,
$$a_n(g(X_n) - g(\mu)) \overset{d}{\longrightarrow} g'(\mu)Z$$







\end{document}