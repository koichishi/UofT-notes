\documentclass[11pt]{article}

% Libraries.
\usepackage{amsmath}
\usepackage{amssymb}
\usepackage{pgfplots}
\usepackage{graphicx}
\usepackage{enumitem}
\usepackage{hyperref}
\usepackage{fancyhdr}
\usepackage{perpage}
\usepackage{float}
\usepackage{esint}

% Property settings.
\MakePerPage{footnote}
\pagestyle{headings}

% Commands
\newcommand{\ti}[1]{\textit{#1}}
\newcommand{\tb}[1]{\textbf{#1}}
\newcommand{\mb}[1]{\mathbb{#1}}
\newcommand{\bx}[0]{\mathbf{x}}
\newcommand{\bv}[0]{\mathbf{v}}
\newcommand{\bw}[0]{\mathbf{w}}
\newcommand{\real}[0]{\mathbb{R}}
\newcommand{\under}[1]{\underline{#1}}
\newcommand{\proof}[0]{\textit{\underline{proof:} }}
\newcommand{\func}[3]{\tb{#1}: {#2} \rightarrow {#3} }
\newcommand{\vx}[0]{\tb{x}}
\newcommand{\vy}[0]{\tb{y}}
\newcommand{\vz}[0]{\tb{z}}
\newcommand{\vo}[0]{\tb{0}}
\newcommand{\va}[0]{\tb{a}}
\newcommand{\vb}[0]{\tb{b}}
\newcommand{\vc}[0]{\tb{c}}
\newcommand{\ve}[0]{\tb{e}}
\newcommand{\vm}[0]{\tb{m}}
\newcommand{\vh}[0]{\tb{h}}
\newcommand{\vf}[0]{\tb{F}}
\newcommand{\vi}[0]{\tb{i}}
\newcommand{\vj}[0]{\tb{j}}
\newcommand{\vk}[0]{\tb{k}}
\newcommand{\vg}[0]{\tb{G}}
\newcommand{\vn}[0]{\tb{n}}
\newcommand{\vu}[0]{\tb{u}}
\newcommand{\vL}[0]{\tb{L}}
\newcommand{\ff}[0]{\tb{f}}
\newcommand{\fg}[0]{\tb{g}}
\newcommand{\rational}[0]{\mathbb{Q}}
\newcommand{\p}[0]{\partial}
\newcommand{\qed}[0]{$\hfill\blacksquare$}
\newcommand{\qerat}{\tag*{$\blacksquare$}}
\newcommand{\lima}{\underset{\vx \rightarrow \va}{\lim}}
\usepackage{amsmath}% http://ctan.org/pkg/amsmath
\newcommand{\notimplies}{%
  \mathrel{{\ooalign{\hidewidth$\not\phantom{=}$\hidewidth\cr$\implies$}}}}


% Attr.
\title{Methods of Data Analysis \\ -- STA302 Course Notes}
\author{Yuchen Wang}
\date{\today}

\begin{document}
    \maketitle
    \tableofcontents
    \newpage
\section{May 7th - Lecture 1}
\paragraph{Definition 1.1 - Statistical Analysis} 
Data Analysis that relies on Probability theory to account for the variability of the data.
\paragraph{Permutation Test 1.2}
Insert random premise, observe two samples Group A and Group B. \\
If the groups have no effect, all of the permutations are equally likely. \\
We can plot the Permutation Distribution with respect to difference between sample means.
\paragraph{Characteristics of Permutation Test 1.3}
\begin{enumerate}
	\item Involves simple probability theory
	\item distribution-free
	\item listing all the permutation for large dataset is almost impossible
\end{enumerate}

\paragraph{Definition 1.4 - Statistical Significance}
We say a difference is \textbf{statistically significant} if it's less probable than our pre-determined significance level. (when p-value $p <$  significance level $\alpha$)
\paragraph{Definition 1.5 - Significant Effect} We say the groups have a \textbf{significant effect} if it causes the variable of interest to be significantly different.

\section{May 9th - Lecture 2}
\subsection{The basics}
\paragraph{Fact 2.1.1} If $H_0$ is true, the p-value $\sim U(0,1)$ \\\\
\under{\ti{remarks}}: This is saying that if p-value is greater than significance level, then it does not say anything about our confidence, it's just a value. Hard to prove, just take it.
\paragraph{Tradeoff Between Type I and Type II Error}
It's common to fix $\alpha$ (significance level or type-I error) and minimize type-II error.
\subsection{One-way Analysis of variance (ANOVA)}
Suppose the response $Y$ is quantitative and the predictor $X$ is categorical, taking $t$ values or levels denoted $1, \hdots, t$. With the regression model, we assume that the only aspect of the conditional distribution of $Y$, given $X = x$, that changes as $x$ changes, is the mean. \\
Suppose we are interested in assessing whether or not there is a relationship between the response and the predictor. There is no relationship if and only if all the conditional distributions are the same. This is true under our assumptions if and only if all the means are equal.
In our case, one-way ANOVA is an extension of the t-test to 3 or more samples focus analysis on group differences.\\
$H_0$: All groups are the same.
If groups are different, we expect there is a bigger difference between groups (\textcolor{blue}{the group effect}) than within groups (\textcolor{blue}{natural variability of the data}).
\paragraph{Basic Definitions}
Suppose we have $T$ groups and $n_t$ observations for the $t$-th group, and we denote each observation as $y$.
\begin{enumerate}
	\item \under{SST}: This is the sum of the squared deviations between each observation and the overall mean:
	$$SST = \sum_{t=1}^T\sum_{i=1}^{n_t}(y_{i,t}-\bar{y})^2$$
	\item \under{SSE}: This is the sum of the squared deviations between each observation and the mean of the group to which it belongs:
	$$SSE = \sum_{t=1}^T\sum_{i=1}^{n_t}(y_{i,t}-\bar{y_t})^2$$
	\item \under{SSG}: This is the sum of the squared deviations between each group mean and the overall mean:
	$$SSG = \sum_{t=1}^T\sum_{i=1}^{n_t}(\bar{y_t}-\bar{y})^2$$
\end{enumerate}

\paragraph{Sum of Squares Decomposition}
Total sum of squares = Within group sum of squares + Between group sum of squares \\
$$\sum_{t=1}^T\sum_{i=1}^{n_t}(y_{i,t}-\bar{y})^2 = \sum_{t=1}^T\sum_{i=1}^{n_t}(y_{i,t}-\bar{y_t})^2 + \sum_{t=1}^T\sum_{i=1}^{n_t}(\bar{y_t}-\bar{y})^2$$
In shorthand:
$$SST = SSE + SSG$$
\proof\\
add $-\bar{y_t}+\bar{y_t}$ inside the squared error term and everything is just like a short proof in STA261, nothing interesting. \qed
\paragraph{ANOVA}
We want to assess how large is SSG relative to SSE, but it would be hard to establish a distribution for SSG/SSE. Knowing a sum of squares divided by its degrees of freedom has a chi-square distribution, we can conclude that
$$SSG/(T-1) \sim \chi_{T-1}^2, \quad SSE/(n-T) \sim \chi_{n-T}^2$$
\paragraph{Theorem 2.2.1} If $SSG/(T-1) = SSE/(n-T)$, then $\frac{SSG/(T-1)}{SSE/(n-T)} \sim F_{T-1, n-T}$ \\\\
\proof \\
In STA261, we've proven that if $\sigma_x^2 = \sigma_y^2$, then $\frac{\hat{\sigma}_x^2}{\hat{\sigma}_y^2} \sim F_{n-1,n-1}$.\\ Since SSG/(T-1) is an estimation for the variation between groups ($\sigma_T$) and $SSE/(n-T)$ is an estimation for the variation within groups ($\sigma_\varepsilon$), then the result follows. \qed
\paragraph{Remarks 2.2.2} Thus a small p-value indicates theses variances are different, which is evidence for the existence of some group effect.
\paragraph{Theorem 2.2.3 One-way ANOVA Table} if p-value $< \alpha$, we reject $H_0$: groups have no effect. \\
	\begin{tabular}{|c|c|c|c|c|}
		\hline
		Source & Sum of Squares & df & Mean Squares & Test Statistic \\
		\hline
		Between & SSG & $T-1$ & $MSG = \frac{SSG}{T-1}$ & $F = \frac{MSG}{MSE}$ \\
		\hline
		Within & SSE & $n - T$ & $MSE = \frac{SSE}{n - T}$ &\\
		\hline
		Total & $SST$ & $ n -1$ &&\\
		\hline
	\end{tabular}


\end{document}