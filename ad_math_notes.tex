\documentclass[11pt]{article}

% Libraries.
\usepackage{amsmath}
\usepackage{amssymb}
\usepackage{pgfplots}
\usepackage{graphicx}
\usepackage{enumitem}
\usepackage{hyperref}
\usepackage{fancyhdr}
\usepackage{perpage}
\usepackage{float}

% Property settings.
\MakePerPage{footnote}
\pagestyle{fancy}
\lhead{Notes by Y.W.}

% Commands
\newcommand{\ti}[1]{\textit{#1}}
\newcommand{\tb}[1]{\textbf{#1}}
\newcommand{\mb}[1]{\mathbb{#1}}
\newcommand{\under}[1]{\underline{#1}}
\newcommand{\proof}[0]{\textit{\underline{proof:} }}
\newcommand{\litran}[0]{$T: V \rightarrow W$ }
\newcommand{\slitran}[0]{Let $ T: V \rightarrow W$ be a linear transformation }
\newcommand{\mt}[0]{$[T]_\alpha^\beta$ }
\newcommand{\qed}[0]{$\hfill\blacksquare$}
\newcommand{\real}[0]{\mathbb{R}}
\newcommand{\vx}[0]{\tb{x}}
\newcommand{\vy}[0]{\tb{y}}
\newcommand{\vz}[0]{\tb{z}}
\newcommand{\vo}[0]{\tb{0}}
\newcommand{\vu}[0]{\tb{u}}
\newcommand{\vw}[0]{\tb{w}}
\newcommand{\vv}[0]{\tb{v}}
\newcommand{\trans}[3]{{#1}: {#2} \rightarrow {#3}}

% Attr.
\title{Advanced Math Notes}
\author{Yuchen Wang}
\date{\today}

\begin{document}
	\maketitle
	\tableofcontents
	\newpage
	\section{Free parameter}
	A variable in a mathematical model which cannot be predicted precisely or constrained by the model and must be estimated experimentally or theoretically.
	\section{Rectifier}
	\subsection{Definition}
	An activation function defined as the positive part of its argument:
	$$f(x) = \max(0,x)$$
	Also known as: ramp function \\
	A unit employing the rectifier is also called a \tb{rectified linear unit (ReLU)}
	\subsection{Softplus}
	A smooth approximation to the rectifier is the analytic function $$f(x) = \log(1+e^x)$$
	Also known as: SmoothReLU\\
	The derivative of softplus is $$f'(x) = \frac{1}{1+e^{-x}}$$(the logistic function)
	\paragraph{Notes}
	The logistic function is a smooth approximation of the derivative of the rectifier, the \tb{Heaviside step function}
	\subsection{Multivariable Generalization to Softplus}
	LogSumExp with the first argument set to zero
	$$LSE_0^+(x_1,\hdots,x_n) := LSE(0,x_1,\hdots,x_n) = \log(1+e^{x_1} + \hdots + e^{x_n})$$
	\paragraph{Notes}
	The LogSumExp function itself is:
	$$LSE(x_1,\hdots,x_n) = \log(e^{x_1} + \hdots + e^{x_n})$$
	and its gradient is the softmax.\\
	The softmax with the first argument set to zero is the multivariable generalization of the logistic function.
	\section{Softmax Function}
	The softmax function takes an un-normalized vector, and normalizes it into a probability distribution. That is, prior to applying softmax, some vector elements could be negative, or greater than one; and might not sum to 1; but after applying softmax, each element $x_i$ is in the interval $[0,1]$, and $\sum_i x_i = 1$ \\
	$$\sigma:\real^K \rightarrow \{\sigma\in \real^K|\sigma_i>0,\sum_{i=1}^K \sigma_i = 1\}$$
	$$\sigma(\vz)_j = \frac{e^{z_j}}{\sum_{k=1}^K e^{z_k}}$$ for $j = 1,\hdots,K$
\end{document}
